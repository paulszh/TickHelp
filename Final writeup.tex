
%% bare_jrnl_comsoc.tex
%% V1.4b
%% 2015/08/26
%% by Michael Shell
%% see http://www.michaelshell.org/
%% for current contact information.
%%
%% This is a skeleton file demonstrating the use of IEEEtran.cls
%% (requires IEEEtran.cls version 1.8b or later) with an IEEE
%% Communications Society journal paper.
%%
%% Support sites:
%% http://www.michaelshell.org/tex/ieeetran/
%% http://www.ctan.org/pkg/ieeetran
%% and
%% http://www.ieee.org/

%%*************************************************************************
%% Legal Notice:
%% This code is offered as-is without any warranty either expressed or
%% implied; without even the implied warranty of MERCHANTABILITY or
%% FITNESS FOR A PARTICULAR PURPOSE! 
%% User assumes all risk.
%% In no event shall the IEEE or any contributor to this code be liable for
%% any damages or losses, including, but not limited to, incidental,
%% consequential, or any other damages, resulting from the use or misuse
%% of any information contained here.
%%
%% All comments are the opinions of their respective authors and are not
%% necessarily endorsed by the IEEE.
%%
%% This work is distributed under the LaTeX Project Public License (LPPL)
%% ( http://www.latex-project.org/ ) version 1.3, and may be freely used,
%% distributed and modified. A copy of the LPPL, version 1.3, is included
%% in the base LaTeX documentation of all distributions of LaTeX released
%% 2003/12/01 or later.
%% Retain all contribution notices and credits.
%% ** Modified files should be clearly indicated as such, including  **
%% ** renaming them and changing author support contact information. **
%%*************************************************************************


% *** Authors should verify (and, if needed, correct) their LaTeX system  ***
% *** with the testflow diagnostic prior tao trusting their LaTeX platform ***
% *** with production work. The IEEE's font choices and paper sizes can   ***
% *** trigger bugs that do not appear when using other class files.       ***                          ***
% The testflow support page is at:
% http://www.michaelshell.org/tex/testflow/



\documentclass[conference]{IEEEtran}
%
% If IEEEtran.cls has not been installed into the LaTeX system files,
% manually specify the path to it like:
% \documentclass[journal,comsoc]{../sty/IEEEtran}


\usepackage[T1]{fontenc}% optional T1 font encoding
\usepackage{textcomp}

\usepackage[noadjust]{cite}
\usepackage{filecontents}
\usepackage{lipsum}
\begin{filecontents*}{references.bib}
@article{1,
  title={A Study of Ad-Hoc Network: A Review},
  author={Student, Vanita Rani PG and Dhir, Renu},
  journal={International Journal},
  volume={3},
  number={3},
  year={2013}
}
@article{2,
  title={A Study on Mobile Ad-Hock Networks (MANETS)},
  author={Senthilkumar, P and Baskar, M and Saravanan, K},
  journal={JMS},
  number={1},
  year={2011}
}
@article{3,
  title={An Overview of Mobile Ad hoc Network: Application, Challenges and Comparison of Routing Protocols},
  author={Kaur, Sukhpreet and Sharma, Chandan},
  journal={IOSR Journal of Computer Engineering (IOSR-JCE), e-ISSN},
  pages={2278--0661},
  year={2013}
}
%%%% di si ge mei zhao dao bib file
@article{5,
  title={Applications, advantages and challenges of ad hoc networks},
  author={Helen, D and Arivazhagan, D},
  journal={JAIR},
  volume={2},
  number={8},
  pages={453--7},
  year={2014}
}
@inproceedings{6,
  title={From ad hoc networks to ad hoc applications},
  author={Garbinato, Beno{\^\i}t and Rupp, Philippe},
  booktitle={Proceedings of the 7th International Conference on Telecommunications},
  pages={145--149},
  year={2003},
  organization={Citeseer}
}
@article{7,
  title={MANET: history, challenges and applications},
  author={Bang, Ankur O and Ramteke, Prabhakar L},
  journal={International Journal of Application or Innovation in Engineering \& Management (IJAIEM)},
  volume={2},
  number={9},
  pages={249--251},
  year={2013}
}
@article{8,
  title={Mobile ad hoc networking: imperatives and challenges},
  author={Chlamtac, Imrich and Conti, Marco and Liu, Jennifer J-N},
  journal={Ad hoc networks},
  volume={1},
  number={1},
  pages={13--64},
  year={2003},
  publisher={Elsevier}
}
@misc{9, 
title={This Little-Known iOS Feature Will Change the Way We Connect}, 
url={http://www.wired.com/2014/03/apple-multipeer-connectivity/}, 
journal={Wired.com}, publisher={Conde Nast Digital}, 
author={Bonnington, Christina}}
@article{10,
  title={Wireless Mesh Networking},
  author={Morris, Robert}
}
@misc{11, 
title={This App Lets You Chat Even When You Have No Reception}, 
url={http://gizmodo.com/this-app-lets-you-chat-even-when-you-have-no-reception-1550447473/1595340521},
journal={Gizmodo}, 
author={Estes, Adam Clark}, year={2014}}
@article{12,
  title={An Overview of security issues in Mobile Ad hoc Networks},
  author={Agrawal, Vikaram M and Chauhan, Hiral},
  journal={INTERNATIONAL JOURNAL OF COMPUTER ENGINEERING AND SCIENCES},
  volume={1},
  number={1},
  pages={9--17},
  year={2015}
}
@article{13,
  title={A brief overview of ad hoc networks: challenges and directions},
  author={Ramanathan, Ram and Redi, Jason},
  journal={IEEE communications Magazine},
  volume={40},
  number={5},
  pages={20--22},
  year={2002}
}
@inproceedings{14,
  title={A review of multipath routing protocols: From wireless ad hoc to mesh networks},
  author={Tsai, Jack and Moors, Tim},
  booktitle={ACoRN early career researcher workshop on wireless multihop networking},
  volume={30},
  year={2006},
  organization={Citeseer}
}

@article{15,
  title={Position-based multicast routing for mobile ad-hoc networks},
  author={Mauve, Martin and F{\"u}{\ss}ler, Holger and Widmer, J{\"o}rg and Lang, Thomas},
  journal={ACM SIGMOBILE Mobile Computing and Communications Review},
  volume={7},
  number={3},
  pages={53--55},
  year={2003},
  publisher={ACM}
}

@article{16,
  title={Opportunistic networking: data forwarding in disconnected mobile ad hoc networks},
  author={Pelusi, Luciana and Passarella, Andrea and Conti, Marco},
  journal={Communications Magazine, IEEE},
  volume={44},
  number={11},
  pages={134--141},
  year={2006},
  publisher={IEEE}
}
@inproceedings{17,
  title={Routing in ad-hoc networks using minimum connected dominating sets},
  author={Das, Bevan and Bharghavan, Vaduvur},
  booktitle={Communications, 1997. ICC'97 Montreal, Towards the Knowledge Millennium. 1997 IEEE International Conference on},
  volume={1},
  pages={376--380},
  year={1997},
  organization={IEEE}
}
@article{18,
  title={Mobile ad hoc networking: milestones, challenges, and new research directions},
  author={Conti, Marco and Giordano, Stefano},
  journal={Communications Magazine, IEEE},
  volume={52},
  number={1},
  pages={85--96},
  year={2014},
  publisher={IEEE}
}
@book{19,
  title={A survey of mobile ad hoc network routing protocols},
  author={Liu, Changling and Kaiser, J{\"o}rg},
  year={2003},
  publisher={Universit{\"a}t Ulm, Fakult{\"a}t f{\"u}r Informatik.}
}
@article{20,
  title={A survey on mobile ad hoc wireless network},
  author={Sesay, Samba and Yang, Zongkai and He, Jianhua},
  journal={Information Technology Journal},
  volume={3},
  number={2},
  pages={168--175},
  year={2004}
}
\end{filecontents*}




% Some very useful LaTeX packages include:
% (uncomment the ones you want to load)


% *** MISC UTILITY PACKAGES ***
%
%\usepackage{ifpdf}
% Heiko Oberdiek's ifpdf.sty is very useful if you need conditional
% compilation based on whether the output is pdf or dvi.
% usage:
% \ifpdf
%   % pdf code
% \else
%   % dvi code
% \fi
% The latest version of ifpdf.sty can be obtained from:
% http://www.ctan.org/pkg/ifpdf
% Also, note that IEEEtran.cls V1.7 and later provides a builtin
% \ifCLASSINFOpdf conditional that works the same way.
% When switching from latex to pdflatex and vice-versa, the compiler may
% have to be run twice to clear warning/error messages.






% *** CITATION PACKAGES ***
%
%\usepackage[noadjust]{cite}
% cite.sty was written by Donald Arseneau
% V1.6 and later of IEEEtran pre-defines the format of the cite.sty package
% \cite{} output to follow that of the IEEE. Loading the cite package will
% result in citation numbers being automatically sorted and properly
% "compressed/ranged". e.g., , [9], [2], [7], [5], [6] without using
% cite.sty will become [1], [2], [5]--[7], [9] using cite.sty. cite.sty's
% \cite will automatically add leading space, if needed. Use cite.sty's
% noadjust option (cite.sty V3.8 and later) if you want to turn this off
% such as if a citation ever needs to be enclosed in parenthesis.
% cite.sty is already installed on most LaTeX systems. Be sure and use
% version 5.0 (2009-03-20) and later if using hyperref.sty.
% The latest version can be obtained at:
% http://www.ctan.org/pkg/cite
% The documentation is contained in the cite.sty file itself.






% *** GRAPHICS RELATED PACKAGES ***
%
\ifCLASSINFOpdf
 \usepackage{graphicx}
  % declare the path(s) where your graphic files are
  % \graphicspath{{../pdf/}{../jpeg/}}
  % and their extensions so you won't have to specify these with
  % every instance of \includegraphics
  % \DeclareGraphicsExtensions{.pdf,.jpeg,.png}
\else
  % or other class option (dvipsone, dvipdf, if not using dvips). graphicx
  % will default to the driver specified in the system graphics.cfg if no
  % driver is specified.
  % \usepackage[dvips]{graphicx}
  % declare the path(s) where your graphic files are
  % \graphicspath{{../eps/}}
  % and their extensions so you won't have to specify these with
  % every instance of \includegraphics
  % \DeclareGraphicsExtensions{.eps}
\fi
% graphicx was written by David Carlisle and Sebastian Rahtz. It is
% required if you want graphics, photos, etc. graphicx.sty is already
% installed on most LaTeX systems. The latest version and documentation
% can be obtained at: 
% http://www.ctan.org/pkg/graphicx
% Another good source of documentation is "Using Imported Graphics in
% LaTeX2e" by Keith Reckdahl which can be found at:
% http://www.ctan.org/pkg/epslatex
%
% latex, and pdflatex in dvi mode, support graphics in encapsulated
% postscript (.eps) format. pdflatex in pdf mode supports graphics
% in .pdf, .jpeg, .png and .mps (metapost) formats. Users should ensure
% that all non-photo figures use a vector format (.eps, .pdf, .mps) and
% not a bitmapped formats (.jpeg, .png). The IEEE frowns on bitmapped formats
% which can result in "jaggedy"/blurry rendering of lines and letters as
% well as large increases in file sizes.
%
% You can find documentation about the pdfTeX application at:
% http://www.tug.org/applications/pdftex





% *** MATH PACKAGES ***
%
\usepackage{amsmath}
% A popular package from the American Mathematical Society that provides
% many useful and powerful commands for dealing with mathematics.
% Do NOT use the amsbsy package under comsoc mode as that feature is
% already built into the Times Math font (newtxmath, mathtime, etc.).
% 
% Also, note that the amsmath package sets \interdisplaylinepenalty to 10000
% thus preventing page breaks from occurring within multiline equations. Use:
\interdisplaylinepenalty=2500
% after loading amsmath to restore such page breaks as IEEEtran.cls normally
% does. amsmath.sty is already installed on most LaTeX systems. The latest
% version and documentation can be obtained at:
% http://www.ctan.org/pkg/amsmath


% Select a Times math font under comsoc mode or else one will automatically
% be selected for you at the document start. This is required as Communications
% Society journals use a Times, not Computer Modern, math font.
\usepackage[cmintegrals]{newtxmath}
% The freely available newtxmath package was written by Michael Sharpe and
% provides a feature rich Times math font. The cmintegrals option, which is
% the default under IEEEtran, is needed to get the correct style integral
% symbols used in Communications Society journals. Version 1.451, July 28,
% 2015 or later is recommended. Also, do *not* load the newtxtext.sty package
% as doing so would alter the main text font.
% http://www.ctan.org/pkg/newtx
%
% Alternatively, you can use the MathTime commercial fonts if you have them
% installed on your system:
%\usepackage{mtpro2}
%\usepackage{mt11p}
%\usepackage{mathtime}


%\usepackage{bm}
% The bm.sty package was written by David Carlisle and Frank Mittelbach.
% This package provides a \bm{} to produce bold math symbols.
% http://www.ctan.org/pkg/bm





% *** SPECIALIZED LIST PACKAGES ***
%
%\usepackage{algorithmic}
% algorithmic.sty was written by Peter Williams and Rogerio Brito.
% This package provides an algorithmic environment fo describing algorithms.
% You can use the algorithmic environment in-text or within a figure
% environment to provide for a floating algorithm. Do NOT use the algorithm
% floating environment provided by algorithm.sty (by the same authors) or
% algorithm2e.sty (by Christophe Fiorio) as the IEEE does not use dedicated
% algorithm float types and packages that provide these will not provide
% correct IEEE style captions. The latest version and documentation of
% algorithmic.sty can be obtained at:
% http://www.ctan.org/pkg/algorithms
% Also of interest may be the (relatively newer and more customizable)
% algorithmicx.sty package by Szasz Janos:
% http://www.ctan.org/pkg/algorithmicx




% *** ALIGNMENT PACKAGES ***
%
%\usepackage{array}
% Frank Mittelbach's and David Carlisle's array.sty patches and improves
% the standard LaTeX2e array and tabular environments to provide better
% appearance and additional user controls. As the default LaTeX2e table
% generation code is lacking to the point of almost being broken with
% respect to the quality of the end results, all users are strongly
% advised to use an enhanced (at the very least that provided by array.sty)
% set of table tools. array.sty is already installed on most systems. The
% latest version and documentation can be obtained at:
% http://www.ctan.org/pkg/array


% IEEEtran contains the IEEEeqnarray family of commands that can be used to
% generate multiline equations as well as matrices, tables, etc., of high
% quality.




% *** SUBFIGURE PACKAGES ***
%\ifCLASSOPTIONcompsoc
%  \usepackage[caption=false,font=normalsize,labelfont=sf,textfont=sf]{subfig}
%\else
%  \usepackage[caption=false,font=footnotesize]{subfig}
%\fi
% subfig.sty, written by Steven Douglas Cochran, is the modern replacement
% for subfigure.sty, the latter of which is no longer maintained and is
% incompatible with some LaTeX packages including fixltx2e. However,
% subfig.sty requires and automatically loads Axel Sommerfeldt's caption.sty
% which will override IEEEtran.cls' handling of captions and this will result
% in non-IEEE style figure/table captions. To prevent this problem, be sure
% and invoke subfig.sty's "caption=false" package option (available since
% subfig.sty version 1.3, 2005/06/28) as this is will preserve IEEEtran.cls
% handling of captions.
% Note that the Computer Society format requires a larger sans serif font
% than the serif footnote size font used in traditional IEEE formatting
% and thus the need to invoke different subfig.sty package options depending
% on whether compsoc mode has been enabled.
%
% The latest version and documentation of subfig.sty can be obtained at:
% http://www.ctan.org/pkg/subfig




% *** FLOAT PACKAGES ***
%
%\usepackage{fixltx2e}
% fixltx2e, the successor to the earlier fix2col.sty, was written by
% Frank Mittelbach and David Carlisle. This package corrects a few problems
% in the LaTeX2e kernel, the most notable of which is that in current
% LaTeX2e releases, the ordering of single and double column floats is not
% guaranteed to be preserved. Thus, an unpatched LaTeX2e can allow a
% single column figure to be placed prior to an earlier double column
% figure.
% Be aware that LaTeX2e kernels dated 2015 and later have fixltx2e.sty's
% corrections already built into the system in which case a warning will
% be issued if an attempt is made to load fixltx2e.sty as it is no longer
% needed.
% The latest version and documentation can be found at:
% http://www.ctan.org/pkg/fixltx2e


%\usepackage{stfloats}
% stfloats.sty was written by Sigitas Tolusis. This package gives LaTeX2e
% the ability to do double column floats at the bottom of the page as well
% as the top. (e.g., "\begin{figure*}[!b]" is not normally possible in
% LaTeX2e). It also provides a command:
%\fnbelowfloat
% to enable the placement of footnotes below bottom floats (the standard
% LaTeX2e kernel puts them above bottom floats). This is an invasive package
% which rewrites many portions of the LaTeX2e float routines. It may not work
% with other packages that modify the LaTeX2e float routines. The latest
% version and documentation can be obtained at:
% http://www.ctan.org/pkg/stfloats
% Do not use the stfloats baselinefloat ability as the IEEE does not allow
% \baselineskip to stretch. Authors submitting work to the IEEE should note
% that the IEEE rarely uses double column equations and that authors should try
% to avoid such use. Do not be tempted to use the cuted.sty or midfloat.sty
% packages (also by Sigitas Tolusis) as the IEEE does not format its papers in
% such ways.
% Do not attempt to use stfloats with fixltx2e as they are incompatible.
% Instead, use Morten Hogholm'a dblfloatfix which combines the features
% of both fixltx2e and stfloats:
%
% \usepackage{dblfloatfix}
% The latest version can be found at:
% http://www.ctan.org/pkg/dblfloatfix




%\ifCLASSOPTIONcaptionsoff
%  \usepackage[nomarkers]{endfloat}
% \let\MYoriglatexcaption\caption
% \renewcommand{\caption}[2][\relax]{\MYoriglatexcaption[#2]{#2}}
%\fi
% endfloat.sty was written by James Darrell McCauley, Jeff Goldberg and 
% Axel Sommerfeldt. This package may be useful when used in conjunction with 
% IEEEtran.cls'  captionsoff option. Some IEEE journals/societies require that
% submissions have lists of figures/tables at the end of the paper and that
% figures/tables without any captions are placed on a page by themselves at
% the end of the document. If needed, the draftcls IEEEtran class option or
% \CLASSINPUTbaselinestretch interface can be used to increase the line
% spacing as well. Be sure and use the nomarkers option of endfloat to
% prevent endfloat from "marking" where the figures would have been placed
% in the text. The two hack lines of code above are a slight modification of
% that suggested by in the endfloat docs (section 8.4.1) to ensure that
% the full captions always appear in the list of figures/tables - even if
% the user used the short optional argument of \caption[]{}.
% IEEE papers do not typically make use of \caption[]'s optional argument,
% so this should not be an issue. A similar trick can be used to disable
% captions of packages such as subfig.sty that lack options to turn off
% the subcaptions:
% For subfig.sty:
% \let\MYorigsubfloat\subfloat
% \renewcommand{\subfloat}[2][\relax]{\MYorigsubfloat[]{#2}}
% However, the above trick will not work if both optional arguments of
% the \subfloat command are used. Furthermore, there needs to be a
% description of each subfigure *somewhere* and endfloat does not add
% subfigure captions to its list of figures. Thus, the best approach is to
% avoid the use of subfigure captions (many IEEE journals avoid them anyway)
% and instead reference/explain all the subfigures within the main caption.
% The latest version of endfloat.sty and its documentation can obtained at:
% http://www.ctan.org/pkg/endfloat
%
% The IEEEtran \ifCLASSOPTIONcaptionsoff conditional can also be used
% later in the document, say, to conditionally put the References on a 
% page by themselves.




% *** PDF, URL AND HYPERLINK PACKAGES ***
%
%\usepackage{url}
% url.sty was written by Donald Arseneau. It provides better support for
% handling and breaking URLs. url.sty is already installed on most LaTeX
% systems. The latest version and documentation can be obtained at:
% http://www.ctan.org/pkg/url
% Basically, \url{my_url_here}.




% *** Do not adjust lengths that control margins, column widths, etc. ***
% *** Do not use packages that alter fonts (such as pslatex).         ***
% There should be no need to do such things with IEEEtran.cls V1.6 and later.
% (Unless specifically asked to do so by the journal or conference you plan
% to submit to, of course. )


% correct bad hyphenation here
\hyphenation{op-tical net-works semi-conduc-tor}


\begin{document}
%
% paper title
% Titles are generally capitalized except for words such as a, an, and, as,
% at, but, by, for, in, nor, of, on, or, the, to and up, which are usually
% not capitalized unless they are the first or last word of the title.
% Linebreaks \\ can be used within to get better formatting as desired.
% Do not put math or special symbols in the title.
\title{Toward practical implementation of Mobile Ad-hoc Mesh networks\\ 
\large Developing and piloting the next-generation peer-to-peer mobile applications}
%
%
% author names and IEEE memberships
% note positions of commas and nonbreaking spaces ( ~ ) LaTeX will not break
% a structure at a ~ so this keeps an author's name from being broken across
% two lines.
% use \thanks{} to gain access to the first footnote area
% a separate \thanks must be used for each paragraph as LaTeX2e's \thanks
% was not built to handle multiple paragraphs
%

\author{Hongda Xiao, Shiyu Chen, Lihui Lu, Zhouhang Shao\\
\normalsize University of California, San Diego\\
Jacobs School of Engineering\\
La Jolla, CA, U.S.A.}

% note the % following the last \IEEEmembership and also \thanks - 
% these prevent an unwanted space from occurring between the last author name
% and the end of the author line. i.e., if you had this:
% 
% \author{....lastname \thanks{...} \thanks{...} }
%                     ^------------^------------^----Do not want these spaces!
%
% a space would be appended to the last name and could cause every name on that
% line to be shifted left slightly. This is one of those "LaTeX things". For
% instance, "\textbf{A} \textbf{B}" will typeset as "A B" not "AB". To get
% "AB" then you have to do: "\textbf{A}\textbf{B}"
% \thanks is no different in this regard, so shield the last } of each \thanks
% that ends a line with a % and do not let a space in before the next \thanks.
% Spaces after \IEEEmembership other than the last one are OK (and needed) as
% you are supposed to have spaces between the names. For what it is worth,
% this is a minor point as most people would not even notice if the said evil
% space somehow managed to creep in.



% The paper headers
%\markboth{Journal of \LaTeX\ Class Files,~Vol.~14, No.~8, August~2015}%
%{Shell \MakeLowercase{\textit{et al.}}: Bare Demo of IEEEtran.cls for IEEE Communications Society Journals}
% The only time the second header will appear is for the odd numbered pages
% after the title page when using the twoside option.
% 
% *** Note that you probably will NOT want to include the author's ***
% *** name in the headers of peer review papers.                   ***
% You can use \ifCLASSOPTIONpeerreview for conditional compilation here if
% you desire.




% If you want to put a publisher's ID mark on the page you can do it like
% this:
%\IEEEpubid{0000--0000/00\$00.00~\copyright~2015 IEEE}
% Remember, if you use this you must call \IEEEpubidadjcol in the second
% column for its text to clear the IEEEpubid mark.



% use for special paper notices
%\IEEEspecialpapernotice{(Invited Paper)}




% make the title area
\maketitle

% As a general rule, do not put math, special symbols or citations
% in the abstract or keywords.
\begin{abstract}
This paper presents an overview of the current use of ad hoc networks in mobile applications. It summarizes the history and the evolution of forming a decentralized network in which each device connects with each other as independent nodes to form a mesh network. The paper also examines the researches that have been done on the topic and the challenges MANET currently faces. Moreover, due to a lack of existing mobile application that uses MANET, an prototype application that is based on MANET is developed to examine for its potential in more practical uses. The results of this paper are applicable to a wide range of decentralized networking  fields that developers can use as a solid ground to tackle with and improve the networking aspects of their applications.
\end{abstract}

% Note that keywords are not normally used for peerreview papers.
\begin{IEEEkeywords}
Mobile ad hoc network, Routing protocols, Mesh networking, Multipeer Connectivity\end{IEEEkeywords}






% For peer review papers, you can put extra information on the cover
% page as needed:
% \ifCLASSOPTIONpeerreview
% \begin{center} \bfseries EDICS Category: 3-BBND \end{center}
% \fi
%
% For peerreview papers, this IEEEtran command inserts a page break and
% creates the second title. It will be ignored for other modes.
\IEEEpeerreviewmaketitle



\section{Introduction}
% The very first letter is a 2 line initial drop letter followed
% by the rest of the first word in caps.
% 
% form to use if the first word consists of a single letter:
% \IEEEPARstart{A}{demo} file is ....
% 
% form to use if you need the single drop letter followed by
% normal text (unknown if ever used by the IEEE):
% \IEEEPARstart{A}{}demo file is ....
% 
% Some journals put the first two words in caps:
% \IEEEPARstart{T}{his demo} file is ....
% 
% Here we have the typical use of a "T" for an initial drop letter
% and "HIS" in caps to complete the first word.
Nowadays, the proliferation of mobile communication networks has enabled many mobile applications to incorporate innovative networking concepts into practical uses. Among the various networking features, the wireless ad-hoc mesh network has demonstrated its advantages through the use of the decentralized network that does not require any pre-existing infrastructures such as wireless access points (WAP) that connects a routers as a separate device. As early as 2005, the conceptual topic of ad-hoc mesh network has been put into use in the military fields \cite{10}. In mobile mesh network (MANET), each mobile phone acts as a node, or access points that each node connects with its adjacent neighbors to form a self-maintaining and resilient networking system. Aside from the usual needs for traditional centralized network settings, people sometimes need to access the internet even when there is not internet presented. \textbf{Sometimes people do not have internet access due to their physical locations and have the immediate needs for help but do not know who\textquotesingle s nearby or can not physically ask people for help under certain situations.} To account for this problem, our team has developed a prototype app that will alleviate people's needs for emergent helps and works in both online and offline setting.

In current establishment of mobile applications that uses the MANET, the most notable of such is Firechat, which is developed by Open Garden and has been popularly used for civil causes such as the Hong Kong protests in 2015. The application uses the Apple?s Multi-peer Connectivity Frameworks, which is a peer-to-peer that allows users to chat without the need of an actual wifi or cellular connections. While the technical details of the implementation of the framework on lower level network layers are not provided, the documentation has stated that it uses ``...infrastructure Wifi networks, peer-to-peer Wi-Fi, and Bluetooth [MPC Documentation]?? to enable the mesh network system. 

In this paper we propose an application that is similar to \textit{Firechat} and enables the users to communicate with nearby others in both online and offline settings.  The focus of this paper is to explore various potential practical uses of MANET while developing an Multi-peer Connectivity application as well as addressing potential security issues and limitations of the status quo of using MANET for mobile applications. 
% You must have at least 2 lines in the paragraph with the drop letter
% (should never be an issue)
\section{System Architecture}
Comparing to other existing mobile applications in the market, our application provides a unique and powerful experience of enabling both online and offline features and utilized various tools to reach a precise location estimation. \textit{Firechat}, which is the only application that utilizes Apple Connectivity framework and made the headlines, was designed to be a social networking application and was unable to main a sizeable client database. Therefore, in our prototype application, we enabled sign-up features so that the users will be able to use the online mode that our applications provides to maintain a relationship with the helpers.  
%figure 1 here
\begin{figure}[h]
\includegraphics[width=0.48\textwidth]{image07}
\caption{Comparison of TickHelp and other similar applications on market
}
\end{figure}

Our prototype application is designed to be an early product that it implements the idea that seeking and getting help should not be limited by the network constraints of the users. If the user happens to be in an area without cellular network and signals, he or she could still use our application to send out emergency requests to the nearby app users so that they can receive immediate helps. And once the user has the internet, he or she could simply log into the application to maintain.

%figure 2 here
\begin{figure}[h]
\includegraphics[width=0.48\textwidth]{image03}
\caption{Function Diagram of \textit{TickHelp}}
\end{figure}
Our prototype application is designed to have the necessary features of a conventional social-networking application and add the offline feature and also user-ranking features. Once the application launches, the user is able to choose to log in as an existing online user or log-in offline by choosing an unique display name. If the user chooses to log in online, the user would be able to switch among four bars: discover page, friend page, personal page and ranking page. On the discovery page, the user is able to discover the nearby online users or nearby offline users. The user could send a help request to the other users by simply clicking on the displayed names to start a one-to-one chat and could also add the other users as friends. Once being helped, the user could also give the other user a `thumbsUp? to indicate that it has been a positive experience. Once a friend request is sent, on the second tab bar page, the user will be able to confirm the pending friend requests and list the other users as friends. On the third tab page, the user is able to upload and change his or her own profile picture to display. Lastly, the ranking page will list all the users in respect to the scores that they have received. 
%figure 3 here
\begin{figure}[h]
\includegraphics[width=0.48\textwidth]{image01}
\caption{System Architecture of \textit{TickHelp}}
\end{figure}
As indicated in Figure 3, \text{ReadyViewController} will be the initial controller of our application. If the user chooses to login as guest, \text{displayNameController} will be invoked and it will allow the user to enter the name he/she wish to display. After the user clicks on the submit button, \textit{offlineViewController} will be invoked and it list nearby users on the view through Multi-peer Connectivity Framework. When two users choose to chat with each other, a peer-to-peer session will be constructed between them, and our application will move to \textit{offlineChatController}, which will allow them to chat offline. Go back to the \textit{ReadyViewController}. If the user chooses to login, our application will move to \text{LogViewController}; if the user chooses to sign up, our application will move to \textit{LogViewController}. When the submit button on either page is clicked, \textit{tabBarConrtoller} will be invoked. Then the controller to be invoked can be \textit{PersonalPageController}, \textit{RankingViewController}, \textit{SwitchFriendController} or \textit{ConversationListController}. When \textit{ConversationListConrtoller} is invoked, the nearby users will be displayed. The user can either choose to chat with another user online with \textit{onlineChatController} or offline with \textit{onlineChatController}. When clicking on the adding button, user A can add user B as friend through \textit{pendingFriendConrtoller}. When user B accepts the request, both user A and user B can see each other on the contact list with \textit{FriendListController}.

\section{Implementation}
Our application utilizes several software resources. For our online mode, we use GeoFire for localization. With Geofire, our application is able to retrieve the location of users based on their longitude and latitude  so that it can help a user to search for nearby users easily. We use Multi-peer Connectivity Framework to implement chatting and user discovery features in offline. With Multi-peer Connectivity Framework, our application can construct peer-to-peer connection between two users under the same ad-hoc network, so that user can discover and chat with nearby users with bluetooth or Wi-Fi.  Moreover, we utilize several external GitHub libraries to improve the user interface of our application. For example, we use VideoSplashKit to display a short video on our onboarding page, which turns out to makes our application more visually appealing to the users.

\section{Related Works}
\subsection{History and Evolution of MANET}
Ever since the mobile ad-hoc networking concept come into place, researchers and application developers have shown interests in the wide range of usability of MANET technologies. First, an mobile ad-hoc network has demonstrated itself to be self-organizing and adaptive, as well as easy to connect or disconnect from the mesh network at a low cost \cite{7}. Overtime, the life cycles of MANET could be characterized by three generations of development. As early as 1970s, MANET has been used through PRNET (Packet Radio Networks) for military purposes. It was examined for combat abilities through trials. The second generation of the ad-hoc networks dated back to the 1980s, in which time it was used as a packet-switched network without infrastructure to improve radios? performance. The last generation arrives in the 1990s, which the ad-hoc network has been examined for its commercial purposes and the idea of treating devices as a collections of nodes has been introduced in various research conferences \cite{7}.

%figure 4 here

\subsection{MANET Securities and Challenges}
While ad-hoc mobile mesh network has the advantages of no requirement of certain infrastructure sets, self-administration capabilities, and the elimination of costs to set up a network, it still faces some challenges that involves both technical difficulties for its operation and also security concerns [8]. To start, the mobility the individual nodes possess could lead the routing issues \cite{7}. Since the pairs of nodes are allowed to move arbitrarily in the network topologies, and each individual node acts as both the host and the router in the network, it would be difficult to maintain congruent information of all nodes when one of the nodes is removed from the network \cite{12}. Multicast routing, which is the routing protocol to distribute data to multiple recipients, would become more challenging because of the dynamic movements of the individual nodes. In the prototype application that our team developed, the Apple Multi-peer Connectivity SDK also experiences the issue of not being able to deliver the data occasionally. Though the phenomenon is not explicitly addressed in Apple�?s developer guide, it?��s likely the result of packet loss, which is when packets of data failed to reach the destination node during their transmissions, causes bit rate error and interference when nodes are being dynamically reallocated \cite{7}. Moreover, the mobility of the nodes in the network would also give malicious nodes easily access to hinder the data network.
Also, Inter-networking between MANET and an IP based fixed network is also a challenge and requires a more fixed coexisting routing protocols than current mobility management \cite{7}. Since MANET does not have any fixed infrastructure for the nodes to base on to forward messages, the intersection of the fixed network and MANET could be challenging to implement. In our prototype application, the offline mode and the online mode are not compatible with each other because of the conflicts in the routing protocols is expected. To achieve the goal of inter-networking among IP based fixed network and MANET, more advanced routing algorithms that works beyond the current Reactive, Proactive and Hybrid routing protocols are needed. Also, the inter-networking issue of MANET could make it prone to attackers in ways that they could easily reroute or track the movements of each node \cite{2}.
The prior securities concerns raised of MANET could have more serious implications in the matter. Every node in MANET is not safely resided in physically protected spaces and are therefore prone to attacks. The same type of common forms of attacks that have been imposed on fixed networks could also threaten mobile Ad-hoc networks. Attacks on MANET could be characterized as active or passive \cite{8}. The passive attacks only attempts to retrieve information rather than modifying network protocols. It would be extremely hard to detect because it does not modify the network or interrupt the data transmission in any way. However, during active attacks, attackers would reroute and inject information into the MANET network. The malicious node would modify the data stream and generate false routing messages \cite{12}. The attack could take on various forms. The common ones are when the malicious node separates the nodes into two sets and become the intermediate node itself, forcefully reroutes the communication from one node to another wrong node, or obscures the vision of the normal nodes by hiding and misrepresent itself with the network.

%figuressssssssssss

\subsection{Existing applications of MANET}
Even though MANET has a long history of being used on military and ratio softwares, its usage on individual mobile devices for commercial purposes is not common. As priorly stated, the most notable mobile application that uses MANET to enable user communications system is FireChat. During the Hong Kong protests in 2015 and multiple other civil protests occasions, people have used the application to form a peer-to-peer mesh network restricting a central area when the government shuts down the internet. The FireChat application has three communications that take on different forms. It allows people to start a group chat capped at a limit of 1000 people, to join open chat rooms, and initiate conversations with nearby users. However, this app has its drawbacks. 
It?s security features are at a bare minimum as there is no authentication system to check if the same device name belongs to a single user. Also, the chat system is not meant for a private and secure communications and the messages that have been exchanged are prone to be obtained by third parties \cite{4}. 
However, as MANET develops rapidly in the field of mobile computing due to its exceptional feature of requiring low costs to maintain and flexibility, researchers has listed many of its potential applications such as entertainment, sensor network, context aware services and to be embedded on Internet of Things (IoT) \cite{3}.

\section{Summary and Future Work}
The rapid developments in the field of mobile computing has enabled Ad-hoc mobile networking to take on more innovative uses than its current status quo. Its advantages of lacking a fixed infrastructure to form a self-organizing and self-maintaining wireless mesh network enables many potential uses in mobile applications with a low cost. As there are still challenges to the routing protocols and security concerns to MANET, further research for more particular algorithms or particular class of algorithm are needed for MANET to be implemented efficiently on mobile platform applications.

By developing this prototype application, we were able to handle the situation when people need to seek help but not have immediate access to internet connections. We used both online and offline techniques to refine our search features to query the nearby users. For the future strategic development of the application, we plan to further integrate the offline and online features so that the users do not have to access the two network modes separately. By achieving this goal, the user would be able to shorten the time that is needed to send out help requests, which becomes vital given an emergency. We would integrate the settings so that application users could choose whether they would like to receive notifications from the application. If someone sends out the help requests, the requests would quickly appear on the nearby users? screens so that it could be accustomed to quickly. Also, we would add a payment feature that works compatible with Apple pay, Paypal and Venmo and the user could easily update their card informations from the settings page. By doing so, the user could make a quick tip to the helpers as a sign of gratitude and the payment amount would be preset to prevent any excessive amount. Furthermore, we would refine the security features of the application so that the payment information and the user information will be encrypted in a more secure setting. By integrating the aforementioned functionalities, we believe that we would be able to turn our app into a sellable product and truly make an evolutionary change in the application world.


%\hfill mds -- you dui qi
 
%\hfill August 26, 2015


%\subsection{Subsection Heading Here}
%Subsection text here.
%\subsubsection{Subsubsection Heading Here}
%Subsubsection text here.

% needed in second column of first page if using \IEEEpubid
%\IEEEpubidadjcol




% An example of a floating figure using the graphicx package.
% Note that \label must occur AFTER (or within) \caption.
% For figures, \caption should occur after the \includegraphics.
% Note that IEEEtran v1.7 and later has special internal code that
% is designed to preserve the operation of \label within \caption
% even when the captionsoff option is in effect. However, because
% of issues like this, it may be the safest practice to put all your
% \label just after \caption rather than within \caption{}.
%
% Reminder: the "draftcls" or "draftclsnofoot", not "draft", class
% option should be used if it is desired that the figures are to be
% displayed while in draft mode.
%
%\begin{figure}[!t]
%\centering
%\includegraphics[width=2.5in]{myfigure}
% where an .eps filename suffix will be assumed under latex, 
% and a .pdf suffix will be assumed for pdflatex; or what has been declared
% via \DeclareGraphicsExtensions.
%\caption{Simulation results for the network.}
%\label{fig_sim}
%\end{figure}

% Note that the IEEE typically puts floats only at the top, even when this
% results in a large percentage of a column being occupied by floats.


% An example of a double column floating figure using two subfigures.
% (The subfig.sty package must be loaded for this to work.)
% The subfigure \label commands are set within each subfloat command,
% and the \label for the overall figure must come after \caption.
% \hfil is used as a separator to get equal spacing.
% Watch out that the combined width of all the subfigures on a 
% line do not exceed the text width or a line break will occur.
%
%\begin{figure*}[!t]
%\centering
%\subfloat[Case I]{\includegraphics[width=2.5in]{box}%
%\label{fig_first_case}}
%\hfil
%\subfloat[Case II]{\includegraphics[width=2.5in]{box}%
%\label{fig_second_case}}
%\caption{Simulation results for the network.}
%\label{fig_sim}
%\end{figure*}
%
% Note that often IEEE papers with subfigures do not employ subfigure
% captions (using the optional argument to \subfloat[]), but instead will
% reference/describe all of them (a), (b), etc., within the main caption.
% Be aware that for subfig.sty to generate the (a), (b), etc., subfigure
% labels, the optional argument to \subfloat must be present. If a
% subcaption is not desired, just leave its contents blank,
% e.g., \subfloat[].


% An example of a floating table. Note that, for IEEE style tables, the
% \caption command should come BEFORE the table and, given that table
% captions serve much like titles, are usually capitalized except for words
% such as a, an, and, as, at, but, by, for, in, nor, of, on, or, the, to
% and up, which are usually not capitalized unless they are the first or
% last word of the caption. Table text will default to \footnotesize as
% the IEEE normally uses this smaller font for tables.
% The \label must come after \caption as always.
%
%\begin{table}[!t]
%% increase table row spacing, adjust to taste
%\renewcommand{\arraystretch}{1.3}
% if using array.sty, it might be a good idea to tweak the value of
% \extrarowheight as needed to properly center the text within the cells
%\caption{An Example of a Table}
%\label{table_example}
%\centering
%% Some packages, such as MDW tools, offer better commands for making tables
%% than the plain LaTeX2e tabular which is used here.
%\begin{tabular}{|c||c|}
%\hline
%One & Two\\
%\hline
%Three & Four\\
%\hline
%\end{tabular}
%\end{table}


% Note that the IEEE does not put floats in the very first column
% - or typically anywhere on the first page for that matter. Also,
% in-text middle ("here") positioning is typically not used, but it
% is allowed and encouraged for Computer Society conferences (but
% not Computer Society journals). Most IEEE journals/conferences use
% top floats exclusively. 
% Note that, LaTeX2e, unlike IEEE journals/conferences, places
% footnotes above bottom floats. This can be corrected via the
% \fnbelowfloat command of the stfloats package.

% if have a single appendix:
%\appendix[Proof of the Zonklar Equations]
% or
%\appendix  % for no appendix heading
% do not use \section anymore after \appendix, only \section*
% is possibly needed

% use appendices with more than one appendix
% then use \section to start each appendix
% you must declare a \section before using any
% \subsection or using \label (\appendices by itself
% starts a section numbered zero.)
%


%\appendices
%\section{Proof of the First Zonklar Equation}
%Appendix one text goes here.

% you can choose not to have a title for an appendix
% if you want by leaving the argument blank
%\section{}
%Appendix two text goes here.


% use section* for acknowledgment
%\section*{Acknowledgment}


%The authors would like to thank...


% Can use something like this to put references on a page
% by themselves when using endfloat and the captionsoff option.
\ifCLASSOPTIONcaptionsoff
  \newpage
\fi



% trigger a \newpage just before the given reference
% number - used to balance the columns on the last page
% adjust value as needed - may need to be readjusted if
% the document is modified later
%\IEEEtriggeratref{8}
% The "triggered" command can be changed if desired:
%\IEEEtriggercmd{\enlargethispage{-5in}}

% references section

\bibliographystyle{IEEEtran}
\bibliography{references}

\appendix[GitHub Link]
https://github.com/arieeel1110/TickHelp.

% can use a bibliography generated by BibTeX as a .bbl file
% BibTeX documentation can be easily obtained at:
% http://mirror.ctan.org/biblio/bibtex/contrib/doc/
% The IEEEtran BibTeX style support page is at:
% http://www.michaelshell.org/tex/ieeetran/bibtex/
%\bibliographystyle{IEEEtran}
% argument is your BibTeX string definitions and bibliography database(s)
%\bibliography{IEEEabrv,../bib/paper}
%
% <OR> manually copy in the resultant .bbl file
% set second argument of \begin to the number of references
% (used to reserve space for the reference number labels box)

%\begin{thebibliography}{1}

%\bibitem{IEEEhowto:kopka}
%H.~Kopka and P.~W. Daly, \emph{A Guide to \LaTeX}, 3rd~ed.\hskip 1em plus
%  0.5em minus 0.4em\relax Harlow, England: Addison-Wesley, 1999.

%\end{thebibliography}

% biography section
% 
% If you have an EPS/PDF photo (graphicx package needed) extra braces are
% needed around the contents of the optional argument to biography to prevent
% the LaTeX parser from getting confused when it sees the complicated
% \includegraphics command within an optional argument. (You could create
% your own custom macro containing the \includegraphics command to make things
% simpler here.)
%\begin{IEEEbiography}[{\includegraphics[width=1in,height=1.25in,clip,keepaspectratio]{mshell}}]{Michael Shell}
% or if you just want to reserve a space for a photo:

%\begin{IEEEbiography}{Michael Shell}
%Biography text here.
%\end{IEEEbiography}

% if you will not have a photo at all:
%\begin{IEEEbiographynophoto}{John Doe}
%Biography text here.
%\end{IEEEbiographynophoto}

% insert where needed to balance the two columns on the last page with
% biographies
%\newpage

%\begin{IEEEbiographynophoto}{Jane Doe}
%Biography text here.
%\end{IEEEbiographynophoto}

% You can push biographies down or up by placing
% a \vfill before or after them. The appropriate
% use of \vfill depends on what kind of text is
% on the last page and whether or not the columns
% are being equalized.

%\vfill

% Can be used to pull up biographies so that the bottom of the last one
% is flush with the other column.
%\enlargethispage{-5in}



% that?s all folks
\end{document}


